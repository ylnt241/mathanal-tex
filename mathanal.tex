%%%%%%%%%%%%%%%%%%%%%%%%%%%%% Define Article %%%%%%%%%%%%%%%%%%%%%%%%%%%%%%%%%%
\documentclass[14pt]{extreport}
%%%%%%%%%%%%%%%%%%%%%%%%%%%%%%%%%%%%%%%%%%%%%%%%%%%%%%%%%%%%%%%%%%%%%%%%%%%%%%%
\usepackage{hyperref}
\usepackage{iftex}
\iftutex
\usepackage{fontspec}
\defaultfontfeatures{Ligatures={TeX}}
\setmainfont{Noto Serif}
\setsansfont{Noto Sans}
\setmonofont{Noto Sans Mono}
\else
\usepackage[T2A]{fontenc}
\fi
%Hyphenation rules
\usepackage{hyphenat}
\hyphenation{ма-те-ма-ти-ка вос-ста-нав-ли-вать}
\usepackage[english, russian]{babel}
\usepackage{titlesec}
%%%%%%%%%%%%%%%%%%%%%%%%%%%%% Using Packages %%%%%%%%%%%%%%%%%%%%%%%%%%%%%%%%%%
\usepackage[left=3cm, right=1.5cm, top=2cm, bottom=2cm]{geometry}
\usepackage{graphicx}
\usepackage{amssymb}
\usepackage{amsmath}
\usepackage{amsthm}
\usepackage{empheq}
\usepackage{mathtext}
\usepackage{mdframed}
\usepackage{booktabs}
\usepackage{lipsum}
\usepackage{graphicx}
\usepackage{color}
\usepackage{psfrag}
\usepackage{pgfplots}
\usepackage{bm}
%%%%%%%%%%%%%%%%%%%%%%%%%%%%%%%%%%%%%%%%%%%%%%%%%%%%%%%%%%%%%%%%%%%%%%%%%%%%%%%

% Other Settings
\usepackage{setspace}
\onehalfspacing
%%%%%%%%%%%%%%%%%%%%%%%%%% Page Setting %%%%%%%%%%%%%%%%%%%%%%%%%%%%%%%%%%%%%%%
\geometry{a4paper}
\usepackage{parskip}

%%%%%%%%%%%%%%%%%%%%%%%%%% Define some useful colors %%%%%%%%%%%%%%%%%%%%%%%%%%
\definecolor{ocre}{RGB}{243,102,25}
\definecolor{mygray}{RGB}{243,243,244}
\definecolor{deepGreen}{RGB}{26,111,0}
\definecolor{shallowGreen}{RGB}{235,255,255}
\definecolor{deepBlue}{RGB}{61,124,222}
\definecolor{shallowBlue}{RGB}{235,249,255}
%%%%%%%%%%%%%%%%%%%%%%%%%%%%%%%%%%%%%%%%%%%%%%%%%%%%%%%%%%%%%%%%%%%%%%%%%%%%%%%
\titleformat{\chapter}[hang]
{\normalfont\huge\bfseries} 
{\thechapter.}              
{0.5em}                     
{}                          

% Настройка отступов (слева, сверху, снизу)
\titlespacing*{\chapter}{0pt}{-30pt}{20pt}
%%%%%%%%%%%%%%%%%%%%%%%%%% Define an orangebox command %%%%%%%%%%%%%%%%%%%%%%%%
\newcommand\orangebox[1]{\fcolorbox{ocre}{mygray}{\hspace{1em}#1\hspace{1em}}}
%%%%%%%%%%%%%%%%%%%%%%%%%%%%%%%%%%%%%%%%%%%%%%%%%%%%%%%%%%%%%%%%%%%%%%%%%%%%%%%

%%%%%%%%%%%%%%%%%%%%%%%%%%%% English Environments %%%%%%%%%%%%%%%%%%%%%%%%%%%%%
\newtheoremstyle{mytheoremstyle}{3pt}{3pt}{\normalfont}{0cm}{\rmfamily\bfseries}{}{1em}{{\color{black}\thmname{#1}~\thmnumber{#2}}\thmnote{\,--\,#3}}
\newtheoremstyle{myproblemstyle}{3pt}{3pt}{\normalfont}{0cm}{\rmfamily\bfseries}{}{1em}{{\color{black}\thmname{#1}~\thmnumber{#2}}\thmnote{\,--\,#3}}
\theoremstyle{mytheoremstyle}
\newmdtheoremenv[linewidth=1pt,backgroundcolor=shallowGreen,linecolor=deepGreen,leftmargin=0pt,innerleftmargin=20pt,innerrightmargin=20pt,]{theorem}{Theorem}[section]
\theoremstyle{mytheoremstyle}
\newmdtheoremenv[linewidth=1pt,backgroundcolor=shallowBlue,linecolor=deepBlue,leftmargin=0pt,innerleftmargin=20pt,innerrightmargin=20pt,]{definition}{Definition}[section]
\theoremstyle{myproblemstyle}
\newmdtheoremenv[linecolor=black,leftmargin=0pt,innerleftmargin=10pt,innerrightmargin=10pt,]{problem}{Problem}[section]
%%%%%%%%%%%%%%%%%%%%%%%%%%%%%%%%%%%%%%%%%%%%%%%%%%%%%%%%%%%%%%%%%%%%%%%%%%%%%%%

%%%%%%%%%%%%%%%%%%%%%%%%%%%%%%% Plotting Settings %%%%%%%%%%%%%%%%%%%%%%%%%%%%%
\usepgfplotslibrary{colorbrewer}
\pgfplotsset{width=8cm,compat=1.9}
%%%%%%%%%%%%%%%%%%%%%%%%%%%%%%%%%%%%%%%%%%%%%%%%%%%%%%%%%%%%%%%%%%%%%%%%%%%%%%%

%%%%%%%%%%%%%%%%%%%%%%%%%%%%%%% Title & Author %%%%%%%%%%%%%%%%%%%%%%%%%%%%%%%%
\title{МатАнал.\\
Теория на дебильник}
\author{ИИТ}
%%%%%%%%%%%%%%%%%%%%%%%%%%%%%%%%%%%%%%%%%%%%%%%%%%%%%%%%%%%%%%%%%%%%%%%%%%%%%%%

\begin{document}


\maketitle
\newpage

\tableofcontents
\newpage
\sloppy
\chapter{Отображение (область определения, множество значений).}
Отображением $f : x \rightarrow y$ называется правило по которому элементы множества $X$ сопоставляются с элементами множества $Y$.

$$\text{Множество значений: } f(x) = \{y \in Y |  \exists x \in X,  y =  f(x)\}$$
$$\text{Область определений: }D(f) = \{x \in X |  \exists y \in Y,  y =  f(x)\}$$

Отображения равны при условии равенства их области определений и области значений при любых $x$
$$f_1 = f_2,\ если\ D(f_1) = D(f_2) \land \forall x \in D(f) \  f_1(x) = f_2(x)$$

\chapter{Сюръекция.}
\textbf{Сюръекция} - это такое отображение, что каждый элемент области значений имеет хотя бы один прообраз.

\begin{definition}[Сюръекция]
    Отображение называется сюръективным (или сюръекцией, или отображением на ), если
    $$ \forall y \in Y~~\exists x \in X~~~f(x) = y. $$
\end{definition}


Например:
\begin{itemize}
    \item $f: \mathbb{R} \to [-1;1],f(x)=\sin x$~--- сюръективно.
    \item $f:\mathbb{R} \to \mathbb{R}_{\ge 0},f(x)=x^2$~--- сюръективно.
    \item $f:\mathbb{R} \to \mathbb{R},f(x)=x^2$~--- не является сюръективным.
\end{itemize}

\chapter{Инъекция.}
\textbf{Инъекция}~--- это функция, которая переводит разные элементы в разные образы.

\begin{definition}[Инъекция]
    Отображение $f: X \to Y$ называется \textbf{инъекцией} (также \textbf{вложением} или \textbf{отображением в } $Y$), если разные элементы множества $X$ переводятся в разные элементы множества $Y$:
    $$\forall x_1,x_2 \in X~~(x_1 \ne x_2) \Rightarrow (f(x_1) \ne f(x_2)).$$
\end{definition}

Замечания:
\begin{itemize}
    \item Эквивалентно, отображение является инъекций, если:\\
          $
              (f(x_1) = f(x_2)) \Rightarrow (x_1 = x_2)
          $
    \item Отображение $f$ инъективно тогда и только тогда, когда для него существует левое обратное:\\
          $
              \exists g:Y \to X~~g\circ f = {id}_X
          $, где ``$\circ$'' обозначает композицию, а ``${id}_X$''~--- тождественное отображение на $X$
\end{itemize}

Примеры:
\begin{enumerate}
    \item
          $
              f:\mathbb{R}_{>0} \to \mathbb{R}, f(x)=\ln{x}
          $~--- инъективно.
    \item
          $
              f: \mathbb{R}_{\ge 0} \to \mathbb{R}_{\ge 0},f(x)=x^2
          $~--- биективно.
    \item
          $
              f : \mathbb{R}_{\ge 0} \to \mathbb{R}, f(x) = x^2
          $~--- не является инъективным, так как $$f(-2)=f(2)=4.$$
\end{enumerate}

\chapter{Биекция}
\begin{definition}[Биъекция]
    Отображение $f:X \to Y$ называется \textbf{биективным} (или \textbf{биекцией}), если оно инъективно и сюръективно.
\end{definition}

Замечание:\\
Два множества, между которыми существует биекция, называются \textbf{равномощными}.

Свойства:
\begin{itemize}
    \item Отображение $f: X \to Y$ является биективным тогда и только тогда, когда существует обратное отображение $f^{-1} : Y \to X$ такое, что $$ f \circ f^{-1} = id_Y, f^{-1} \circ f = id_X, $$
          где ``id'' обозначает тождественное отображение, а $``\circ''$ композицию функций.
    \item Пусть даны два отображения:\\
          $f: X \to Y$ и $g: Y \to Z$, а $h = g \circ f : X \to Z$~--- их композиция. Тогда $h$ биективно тогда и только тогда, когда $f$ инъективно, а $g$ сюръективно.
    \item В частности, композиция двух биективных отображений сама биективна. Обратное, вообще говоря, неверно.
\end{itemize}
\bigskip
Примеры:
\begin{enumerate}
    \item $id: X \to X$~--- функция, сохраняющая все элементы множества $X$, биективна на этом множестве
    \item $f(x) = x, f(x) = x^3$~--- биективные функции из $\mathbb{R}$ в себя. Вообще, любой одночлен одной переменной нечётной степени является биекцией.
    \item $f(x)=e^x$~---биективная функциия $\mathbb{R}$ в $\mathbb{R}_{+} = (0, \infty)$. Но если её рассматривать как функцию в $\mathbb{R}$, то она уже не будет биективной (у отрицательных чисел не будет прообразов).
    \item $f(x)= \sin{x}$ не является биективной функцией, если считать её определённой на всём $\mathbb{R}$.
\end{enumerate}
\chapter{Обратное отображение.}

Прямое отображение $f : A \to B$ обратимо $f^{-1} : B \to A$ тогда и только тогда, когда оно взаимно однозначно. Откуда следует, что $f$~--- это биективная функция.

На уровне обывательского понимания обратимость означает возможность ``всё вернуть в исходное состояние'', причём, идеально и однозначно.

Так, если мы умножаем число ``икс'' на два: $y = f(x) = 2x$, то всегда можно взять ``игрек'', разделить его пополам и получить исходное число: $ x = f^{-1}(y) = \frac{y}{2} $.

\chapter{Композиция двух отображений.}
Пусть $f: x \rightarrow y ,\  g: y \rightarrow z$ , тогда $g\circ f : x \rightarrow z$ - композиция.по правилу $g(f(x))$


\textbf{Если же обе функции $f$ и $g$ биективны, то их композиция $gf$~--- тоже биекция}.

К примеру, увеличим ``икс'' на три $f(x) = x + 3$, а затем возведём число ``<$e$''> в эту степень: $ y = g(f(x)) = e^{x+3} $~--- в результате получена взаимно однозначная функция, поскольку функции $f$ и $g$, очевидно, биективны. Более того, композиция биективных функций обратима: всегда и однозначно можно выяснить исходное значение ``<икс''>: $ x = f^{-1}(g^{-1}(y)) = \ln{y-3} $.

\textbf{И другой важный факт}: в общем случае композиция не перестановочна $ gf \ne fg $. Так, если ``<икс''> сначала возвести в квадрат, а затем найти синус, то получится функция $ y = sin{x^2} $. Если же сначала найти синус икс, а затем возвести результат в квадрат, то получится другая функция: $ y = \sin^2{x} $.

Композиционные функции называются \textbf{сложными}. Образно говоря, в композициях одна функция вложена в другую.

\chapter{Принцип математической индукции.}
$$Если\ \mathbb{E}\subset \mathbb{N} \land 1 \in \mathbb{E}\ \land\ (x \in \mathbb{E} \implies(x + 1) \in \mathbb{E}), \implies \mathbb{E} = \mathbb{N} $$
\chapter{Определения множества натуральных чисел}
Множество $X$ для которого выполнено $\forall x \in X: (x + 1) \in X$ будем называть \textbf{индуктивным}

Множество $\mathbb{N}$ - \textbf{наименьшее индуктивное множество}, содержащее единицу.

\chapter{Счётное и несчётное множество.}
$\mathbb{Q}$~--- множество рациональных чисел. $\mathbb{Q} = \frac{m}{n}, m \in \mathbb{Z}, n \in \mathbb{Z}\setminus 0$

Счётное множество~--- это бексонечное множество, элементы которого возможно пронумеровать всеми натуральными числами. Или более формально:
\begin{center}
    Множество $X$ является счётным, если существует биекция со множеством натуральных чисел: $X \leftrightarrow N$ или множество, равномощное множеству натуральных чисел.
\end{center}
Множество $X$ равномощно множеству $Y$, если $\exists$ биективное отображение: $ f : x \to y$

\chapter{Интервал, отрезок, промежуток.}
$(a;b)$ - интервал
$$(a;b) = \{x \in \mathbb{R}: a < x < b \}$$
$[a;b]$ - отрезок
$$[a;b] = \{x \in \mathbb{R}: a \leq x \leq b \}$$
$(a;b]$ — левый полуинтервал
$$(a;b] = \{x\in \mathbb{R}: a < x \leq b\}$$
Аналогично определяется правый

$[a, +\infty)$ - бесконечный промежуток

$$[a, +\infty = \{x \in \mathbb{R} : x\geq a\})$$
Аналогично определяются другие бесконечные промежутки

Точка $x_0$ называется **предельной точкой множества** $X$, если $\forall\ \delta > 0\ \exists\ x \in {X}, х \in O_\delta(x_0)$
\chapter{Определения окрестности.}
Интервал $(a;b)$, где $x_0 \in (a;b)$ называется окрестностью точки $x_0$ обозначается $O(x_0)$

\chapter{Определение проколотой окрестности.}

Интервал $(a;x_0)\land (x_0;b)$ называется проколотой окрестностью точки $x_0$. Обозначается $\mathbb{O}(x_0)$
%%(Тут типа о внутри О)%%

\chapter{Определение внутренней точки множества.}

Точка $x_0$ называется \textbf{внутренней точкой множества} $X$, если $\exists \delta > 0, x_0 \in {X} \implies O_\delta(x_0) \subset \mathbb{X}$

\chapter{Определение предельной точки множества.}

Точка $x_0$ называется \textbf{предельной точкой множества} $X$, если $\forall\ \delta > 0\ \exists\ x \in {X}, х \in O_\delta(x_0)$

В любой окрестности $x_0$ есть хотя бы одна точка из X

\chapter{Определение открытого множества.}
Множество $X$ называется **открытым**, если $x \in X \implies (\exists\ \delta > 0, O_\delta(x) \subset {X}$

\chapter{Определение замкнутого множества.}
$X$, если $\mathbb{R}\backslash {X}$ открыто

\chapter{Верхняя$\setminus$нижняя грань множества.}

Точка $x_0$ называется максимальным элементом ${X}$ ($maxX$) , если 


$x_0 \in {X} \land (\forall x \in {X}, x \leq x_0)$


Точка $x_0$ называется минимальным элементом ${X}$ ($minX$), если 


$x_0 \in {X} \land (\forall x \in {X}, x \geq x_0)$
\chapter{Ограниченное снизу$\setminus$сверху множество. Ограниченное множество.}

$\mathbb{X}$ называется \textbf{ограниченным сверху(снизу)}, если:
$$\exists\ {M} \in \mathbb{R}:\forall\ x \in {X},\ x \leq \mathbb{M}\ (x \geq \mathbb{M})$$
$\mathbb{X}$ называется \textbf{ограниченным}, если оно ограниченно сверху или снизу

\chapter{Определение последовательности вложенных отрезков.}
пусть $\{x_n\}$ - последовательность множеств

тогда ${x_n}$ - последовательность вложенных отрезков, если $\forall n \ x_n\supset x_{n+1}$ 
\chapter{Принцип вложенных отрезков Коши-Кантора (формулировка)}

$\forall\{x_n\}$ вложенных отрезков $\exists c\in \mathbb{R}$ - общая точка. Если при этом $\forall \epsilon > 0 \ \exists N \in \mathbb{N} \ \forall n > N$


$|I_n| = |b_n - a_n| < \epsilon$, тогда с - единственная.

Если длины отрезков стремятся к нулю:
$$
    \lim_{n \rightarrow \infty}(b_n - a_n) = 0
$$
то такая точка единственная.

\chapter{Определения покрытия и подпокрытия множества.}

Система $S=\{x\}$ покрывает множество $Y$, если $\forall y \in Y \ \exists X\subset S: y\in X$, то есть $Y\subset \bigcup_{x\subset S} X$ 

\chapter{Определение предела числовой последовательности}
A, если $$\forall\ \epsilon > 0\ \exists\ N(\epsilon) \in \mathbb{N}, n > N(\epsilon) \ |a_n - A| < \epsilon$$

\chapter{Определение ограниченной последовательности.}


 {$a_n$} - называется ограниченной, если $\exists\ M \in \mathbb{R}:\forall\ n \in \mathbb{N},\ |a_n| < M$

\chapter{Определение фундаментальной последовательности.}

 {$a_n$} - фундаментальная, если $\forall \varepsilon > 0\ \exists\  N \in \mathbb{N}: \forall\ n,m > N,\ |a_n - a_m| < \varepsilon$

\chapter{Критерий Коши для последовательностей.}
 {$a_n$} сходящаяся $\iff$ {$a_n$} фундаментальная

\chapter{Определения монотонных последовательностей.}

 {$a_n$} называется монотонно - возрастающей (убывающей), если
$\forall\  n,\ a_n < a_{n+1}\  (a_n > a_{n + 1})$

\chapter{Число е (определение).}
$$lim_{n \to \infty} (1 + \frac{1}{n})^n = e$$

\chapter{Числовой ряд, частичная сумма, сумма сходящегося ряда.}
Пусть {$a_n$} - числовая последовательность
$S_1 = a_1, S_2 = a_1 + a_2, S_3 = a_1 + a_2 + a_3$
$S_k = a_1+...+a_k$
{$S_k$ } - последовательность частичных сумм
Числовой ряд:
$$\sum_{n = 1}^\infty a_n$$ $a_n$ - общий член ряда

Числовой ряд $\sum_{n = 1}^\infty a_n$ называется сходящимся, если $\exists\ конечный\ lim_{k \to \infty}S_k = S,\ тогда\ S = \sum_{n = 1}^\infty a_n$

\chapter{Необходимый признак сходимости числового ряда.}
Числовой ряд $$\sum_{n = 1}^\infty a_n \ сходится\iff\forall\ \varepsilon > 0\  \exists\ N \in \mathbb{N} :\forall\ m > k> N |\sum_{n = k}^m a_n| < \varepsilon $$

\chapter{Признаки сравнения (формулировка).}
Первый признак сравнения
Пусть $$\exists\ N \in \mathbb{N} : \forall\ n > N :a_n \leq b_n$$
Тогда если $\sum_{n = 1}^\infty b_n$ сходится, то  $\sum_{n = 1}^\infty a_n$ сходится
Если $\sum_{n = 1}^\infty a_n$ расходится, то $\sum_{n = 1}^\infty b_n$ расходится

\textbf{Интегральный признак}
$$\forall\ n \in \mathbb{N}\ a_n \geq 0,\ lim_{n \to \infty}a_n = 0,\ тогда \sum_{n = 1}^\infty a_n\ и \int_1^{+\infty} a_n dn\ сходится\ и \ расходится\ одновременно$$

\textbf{Второй признак сравнения}
Пусть $a_n\sim b_n, \ n \rightarrow \infty$
То есть $lim_{n \to \infty}\frac{a_n}{b_n} = 1,\ тогда\ \sum_{n = 1}^\infty a_n \ и\ \sum_{n = 1}^\infty b_n$ сходится и расходится одновременно

\chapter{Признак Даламбера (формулировка).}
Пусть $lim_{n \to \infty}\frac{a_{n+1}}{a_n} = q$
Если $q < 1$ ряд сходится
Если $q > 1$ ряд расходится
Если $q = 1$ признак не работает

\chapter{Радикальный признак Коши (формулировка).}
Пусть $lim_{n \to \infty} \sqrt[n]{a_n} = q$
Если  $q < 1$ ряд сходится
Если $q > 1$ ряд расходится
Если $q = 1$ признак не работает, но если последовательность стремится к своему пределу $q$ сверху, то ряд сходится.

\chapter{Признак Лейбница (формулировка).}
Признак Лейбница~--- признак сходимости знакочередующегося ряда, установлен Готфридом Лейбницем.

\textbf{Формулировка теоремы}

Пусть дан знакочередующийся ряд
$$
S = \sum\limits_{n=1}^{\infty} (-1)^{n-1}\cdot b_n, b_n \ge 0,
$$

для которого выполняются следующие условия:
\begin{enumerate}
    \item $b_n \ge b_{n+1}$, начиная с некоторого номера ($ n \ge N$),
    \item $\lim_{n \to \infty} b_n = 0$
\end{enumerate}

Тогда такой ряд сходится

\chapter{Определения абсолютно и условно сходящихся рядов.}

Если числовой ряд сходится абсолютно, то он сходится

\chapter{Определение предела функции в точке по Гейне.}

$$\forall\ \{x_n\} \in \mathbb{X}\ lim_{n\to\infty}x_n = a\implies lim_{n\to\infty}f(x_n) = A$$

\chapter{Определение предела функции в точке по Коши}

$$\forall\ \varepsilon > 0,\exists\ \delta > 0, \forall\ x \in \mathbb{X}\ 0<|x-a| < \delta \implies |f(x) - A| < \varepsilon$$

\chapter{Первый замечательный предел}
$$lim_{x \to 0} \frac{sin(x)}{x} = 1$$
В качестве числителя и знаменателя могут быть любые эквивалентные функции:
$$x \sim sin(x) \sim tg(x) \sim arcsin(x) \sim arctg(x) \sim ln(x + 1) \sim e^x - 1 \sim \frac{a^x - 1}{ln(a)}$$
А также
$$1-cos(x) \sim \frac{x^2}{2}$$

\chapter{Второй замечательный предел}

$$lim_{x \to \infty}(1 + \frac{1}{x})^x = e$$

\chapter{Определение эквивалентных функций.}
\begin{enumerate}
    \item $f(x) \sim f(x)\text{ при базе } \beta$
    \item $f(x) \sim g(x) \iff g(x) \sim f(x)\text{ при базе }\beta$
    \item $(f(x) \sim g(x))\ \land (g(x) \sim h(x)) \implies f(x)\sim h(x)$
\end{enumerate}

\chapter{Примеры эквивалентных функций (таблица)}
$$x \sim sin(x) \sim tg(x) \sim arcsin(x) \sim arctg(x) \sim ln(x + 1) \sim e^x - 1 \sim \frac{a^x - 1}{ln(a)}$$
А также
$$1-cos(x) \sim \frac{x^2}{2}$$

\chapter{Определение непрерывной функции в точке.}
$\lim_{x\to a-}f(x)=\lim_{x\to a+}f(x) = f(a) \implies f(x)$непрерывна в точке $x=a$ 
\chapter{Определение непрерывной функции в точке через односторонние пределы.}
Функция $f(x)$ называется \textbf{непрерывной справа (слева) в точке $X_0$}, если она определена на некоторой правосторонней (левосторонней) окрестности $U(X_0 + 0)~~(U(X_0-0))$ этой точки, и если правый (левый) предел в точке $X_0$ равен значению функции в $X_0$:
$$
    f(X_0 + 0) \equiv lim_{X \rightarrow X_0 + 0} f(X) = f(X_0)
$$
$$
    f(X_0 - 0) \equiv lim_{X \rightarrow X_0 - 0} f(X) = f(X_0)
$$
\chapter{Определения точек устранимого и неустранимого разрыва.}
Точка устранимого разрыва:

Если $\exists$ конечные $\lim_{x\to a-}f(x) = \lim_{x\to a+}f(x) \neq f(a) \implies \ x=a$ - точка устранимого разрыва $y=f(x)$

Точка неустранимого разрыва:

Если $\exists$ конечные

$\lim_{x\to a-}f(x) \neq \lim_{x\to a+}f(x)$

\chapter{Определение точек разрыва первого и второго рода}
если $\not{\exists}$ или равен $\infty,\pm\infty \lim_{x\to a-}f(x)$ или $\lim_{x\to a+}f(x)$, то x = a - точка разрыва второго рода.

\chapter{Определение производной функции в точке.}
$y=f(x_0)+f'(x_0)(x-x_0)$, где $x_0$ - точка касания

\chapter{Определение дифференцируемой функции в точке.}

Функция $f:\mathbb{E} \to \mathbb{R}$ называется дифференцируемой в точке $x_0 \in \mathbb{E}$ , если $\exists\ A: f(x) - f(x_0) =A(x-x_0) + o*(x-x_0)$ , если $x \to x_0,\ x\in\mathbb{E}$
При этом $$f(x) = o(x-x_0),x\to0$$
А также $$lim_{x \to 0}\frac{f(x)}{x-x_0} = 0$$

\chapter{Критерий дифференцируемости функции.}
Функция $f$ дифференцируема в точке $x_0 \iff$ когда существует конечный $lim_{x\to x_0}\frac{\Delta f}{\Delta x} = A(x_0) = f(x_0)$

\chapter{Уравнение касательной к графику функции.}
$$y = f(x_0)+f'(x_0)(x-x_0)$$

\chapter{Теорема о производной композиции (формулировка).}
$f$ дифференцируема в $x_0$, $g$ - дифференцируема в $y_0 = f(x_0)$

Тогда $f\circ g$ дифференцируема в $x_0$, причём 

$(g\circ f)'(x_0)=g'(f(x_0))*f'(x_0)$

\chapter{Таблица производных основных элементарных функций.}
\renewcommand{\arraystretch}{2} % Увеличиваем межстрочный интервал для дробей
\begin{center}
    \begin{tabular}{|c|c|}
        \hline
        \textbf{Функция} $f(x)$ & \textbf{Производная} $f'(x)$ \\ \hline
        $C$                     & $0$                          \\ \hline
        $x^n$                   & $n x^{n-1}$                  \\ \hline
        $\sqrt{x}$              & $\dfrac{1}{2\sqrt{x}}$       \\ \hline
        $e^x$                   & $e^x$                        \\ \hline
        $a^x$                   & $a^x \ln a$                  \\ \hline
        $\ln x$                 & $\dfrac{1}{x}$               \\ \hline
        $\sin x$                & $\cos x$                     \\ \hline
        $\cos x$                & $-\sin x$                    \\ \hline
        $\text{tg}\, x$         & $\dfrac{1}{\cos^2 x}$        \\ \hline
        $\text{ctg}\, x$        & $-\dfrac{1}{\sin^2 x}$       \\ \hline
        $\arcsin x$             & $\dfrac{1}{\sqrt{1-x^2}}$    \\ \hline
        $\text{arctg}\, x$      & $\dfrac{1}{1+x^2}$           \\ \hline
    \end{tabular}
\end{center}

\chapter{Определение точки локального экстремума.}
Точка локального экстремума называется точкой локального максимума или минимума

Точка $x_0$ называется точкой локального максимума (минимума) функции $f(x)$, если $\exists\ \delta > 0 : \forall x \in \mathbb{O}_\delta(x_0),\ f(x) \leq f(x_0) \ [f(x) \geq f(x_0)]$

Точка $x_0$ называется точкой строгого локального максимума $[минимума]$, если в определении выше  заменить $\leq$ на < и $\geq$ на >.

$f(x_0)$ также называется точкой локального максимума (минимума)

\chapter{Определение локального экстремума.}
Выше.


\chapter{Определение точки внутреннего локального экстремума.}
Если $x_0$ предельная точка из множества локальных минимумов и локальных максимумов, тогда $x_0$ называется точкой внутреннего локального множества

\chapter{Теорема Ферма о локальном экстремуме (формулировка).}
$x_0$ - точка внутреннего локального экстремума множества $\mathbb{E}$ функции $f:\mathbb{E} \to \mathbb{R}$
$f(x)$ непрерывна на $\mathbb{E}$ и дифференцируема в точке $x_0$
Тогда $$f'(x_0) = 0$$

\chapter{Теорема Ролля (формулировка).}
$f: [a,b] \to \mathbb{R}$
Пусть функция $f(x)$ удовлетворяет трем условиям:
1. \textbf{Непрерывна} на отрезке $[a, b]$
2. \textbf{Дифференцируема} на интервале $(a, b)$
3. \textbf{Принимает равные значения} на концах отрезка: $f(a) = f(b)$
Тогда существует точка $\xi \in (a, b)$ такая, что:
$$f'(\xi) = 0$$

\chapter{Формула конечных приращений Лагранжа.}
Пусть $f:[a,b]\to \mathbb{R}$ 
$f(x)$ удовлетворяет условиям:
\begin{itemize}
    \item \textbf{Непрерывна} на отрезке $[a, b]$
    \item \textbf{Дифференцируема} на интервале $(a, b)$
\end{itemize}


Тогда $\exists\xi \in (a, b)$ такая, что:
$$f'(\xi) = \frac{f(b) - f(a)}{b - a}$$

\chapter{Формула Тейлора.}
$$f(x) = \sum_{n=0}^{\infty} \frac{f^{(n)}(a)}{n!} (x-a)^n$$

\chapter{Разложение основных элементарных функций по формуле Тейлора.}
\textbf{Для $e^x$}
$$e^x = \sum_{n=0}^{\infty} \frac{x^n}{n!} = 1 + x + \frac{x^2}{2!} + \frac{x^3}{3!} + \frac{x^4}{4!} + O_{x \to 0}(x^n)$$
\textbf{Для $sin(x)$}
$$\sin x = \sum_{n=0}^{\infty} \frac{(-1)^n x^{2n+1}}{(2n+1)!} = x - \frac{x^3}{3!} + \frac{x^5}{5!} - \frac{x^7}{7!} + O_{x \to 0}(x^n)$$
\textbf{Для $cos(x)$}
$$\cos x = \sum_{n=0}^{\infty} \frac{(-1)^n x^{2n}}{(2n)!} = 1 - \frac{x^2}{2!} + \frac{x^4}{4!} - \frac{x^6}{6!} + O_{x \to 0}(x^n)$$
\textbf{Для $\frac{1}{1-x}$ Геометрическая прогрессия}
$$\frac{1}{1-x} = \sum_{n=0}^{\infty} x^n = 1 + x + x^2 + x^3 + x^4  + O_{x \to 0}(x^n)$$
\textbf{Для $(1+x)^\alpha$ Биномиальный ряд}
$$(1+x)^\alpha = 1 + \alpha x + \frac{\alpha(\alpha-1)}{2!}x^2 + \frac{\alpha(\alpha-1)(\alpha-2)}{3!}x^3  + O_{x \to 0}(x^n)$$

\chapter{Определение точки перегиба}
x = a, если

$\exists \delta >0$ такая что на интервалах $(a-\delta, a)$ и $(a, a+\delta)$ функция выпукла в разные стороны.
\chapter{Теорема об экстремумах дифференцируемой функции (формулировка).}
Если точка $x = x_0$ является точкой строгого или нестрогого локального экстремума функции $f(x)$ и она дифференцируема в этой точке, то её производная в $x_0$ равна нулю:
$$
    f'(x_0)=0.
$$

\chapter{Теорема об экстремумах n раз дифференцируемой функции (формулировка).}
TODO

\chapter{Определение выпуклой (строго) функций.}
Пусть $f:(a,b) \to \mathbb{R}$ дифференцируема на $(a,b)$
Тогда $f(x)$ выпукла на $(a,b)$ и $[a,b]$ $\iff \forall\ x \in (a,b) f"(x) < 0$

\chapter{Определение вогнутой (строго) функций.}

Пусть $f:(a,b) \to \mathbb{R}$ дифференцируема на $(a,b)$
Тогда $f(x)$ выпукла на $(a,b)$ и $[a,b]$ $\iff \forall\ x \in (a,b) f"(x) > 0$

\chapter{Определение точки перегиба.}
Точка $x = a$ называется точкой перегиба функции $f(x)$ если существует такое $\delta > 0$, что в интервалах ($a-\delta,~a$) и ($a;~a+\delta$) эта функция выпукла в разные стороны.

\chapter{Теорема о связи выпуклой и строго выпуклой функции с первой производной}

\begin{theorem}
    Пусть функция $f(x)$ дифференцируема на интервале $(a, b)$. Тогда:
    \begin{enumerate}
        \item $f(x)$ является \textbf{выпуклой вниз} на $(a, b)$ тогда и только тогда, когда её производная $f'(x)$ является неубывающей на этом интервале.
        \item $f(x)$ является \textbf{строго выпуклой вниз} на $(a, b)$ тогда и только тогда, когда её производная $f'(x)$ является строго возрастающей на этом интервале.
    \end{enumerate}
\end{theorem}

\chapter{Следствие о связи выпуклой и строго выпуклой функции со второй производной}


Пусть функция $f(x)$ дважды дифференцируема на интервале $(a, b)$. Тогда:
\begin{enumerate}
    \item Если $f''(x) \ge 0$ для всех $x \in (a, b)$, то функция $f(x)$ выпукла вниз на этом интервале.
    \item Если $f''(x) > 0$ для всех $x \in (a, b)$ (за исключением, возможно, конечного числа точек), то функция $f(x)$ является строго выпуклой вниз на этом интервале.
\end{enumerate}


\chapter{Формулировка теоремы о достаточном условии}

\begin{theorem}[Достаточное условие выпуклости]
    Если функция $f(x)$ непрерывна на отрезке $[a, b]$ и внутри него имеет вторую производную, которая сохраняет знак, то:
    \begin{itemize}
        \item При $f''(x) > 0$ график функции направлен выпуклостью вниз (функция выпукла).
        \item При $f''(x) < 0$ график функции направлен выпуклостью вверх (функция вогнута).
    \end{itemize}
\end{theorem}

\chapter{Примеры на все случаи}

\textbf{Строгая выпуклость вниз}
Рассмотрим функцию $f(x) = e^x$.
Производные: $f'(x) = e^x$, $f''(x) = e^x$.
Так как $e^x > 0$ для любого $x$, функция строго возрастает и строго выпукла вниз на всей числовой прямой.


\textbf{Выпуклость, но не строгая}
Рассмотрим линейную функцию $f(x) = kx + b$.
Производные: $f'(x) = k$, $f''(x) = 0$.
Так как $f''(x) \ge 0$ выполняется, функция является выпуклой (и одновременно вогнутой), но не строго.


\textbf{Смена характера выпуклости}
Рассмотрим $f(x) = x^3$.
Вторая производная: $f''(x) = 6x$.
\begin{itemize}
    \item При $x > 0 \implies f''(x) > 0$ (выпукла вниз).
    \item При $x < 0 \implies f''(x) < 0$ (выпукла вверх/вогнута).
    \item Точка $x=0$ является точкой перегиба.
\end{itemize}


\chapter{Определение первообразной.}
Функция $F(x)$ называется первообразной функции $f(x)$ , если
$$dF(x) = f(x)dx,\ то\ есть f(x) = F'(x)$$

\chapter{Определение неопределенного интеграла.}
Совокупность всех первообразных функции, отличных на константу интегрирования C.

$$\int f(x)dx=F(x)+C$$

\chapter{Таблица неопределенных интегралов.}
На след.странице.
\newpage
\begin{center}
    \begin{tabular}{|c|c|}
        \hline
        \textbf{№} & \textbf{Интеграл}                                                                                  \\ \hline
        1          & $\displaystyle \int 0 \cdot dx = C$                                                                \\ \hline
        2          & $\displaystyle \int 1 \cdot dx = x + C$                                                            \\ \hline
        3          & $\displaystyle \int x^n dx = \dfrac{x^{n+1}}{n+1} + C \quad (n \neq -1)$                           \\ \hline
        4          & $\displaystyle \int \dfrac{dx}{x} = \ln |x| + C$                                                   \\ \hline
        5          & $\displaystyle \int e^x dx = e^x + C$                                                              \\ \hline
        6          & $\displaystyle \int a^x dx = \dfrac{a^x}{\ln a} + C$                                               \\ \hline
        7          & $\displaystyle \int \sin x \, dx = -\cos x + C$                                                    \\ \hline
        8          & $\displaystyle \int \cos x \, dx = \sin x + C$                                                     \\ \hline
        9          & $\displaystyle \int \dfrac{dx}{\cos^2 x} = \text{tg}\, x + C$                                      \\ \hline
        10         & $\displaystyle \int \dfrac{dx}{\sin^2 x} = -\text{ctg}\, x + C$                                    \\ \hline
        11         & $\displaystyle \int \dfrac{dx}{1+x^2} = \text{arctg}\, x + C$                                      \\ \hline
        12         & $\displaystyle \int \dfrac{dx}{\sqrt{1-x^2}} = \arcsin x + C$                                      \\ \hline
        13         & $\displaystyle \int \dfrac{dx}{x^2 - a^2} = \dfrac{1}{2a} \ln \left| \dfrac{x-a}{x+a} \right| + C$ \\ \hline
        14         & $\displaystyle \int \dfrac{dx}{\sqrt{x^2 \pm k}} = \ln |x + \sqrt{x^2 \pm k}| + C$                 \\ \hline
    \end{tabular}
\end{center}
\chapter{Теорема о методе замены переменной в неопределенном интеграле (формулировка).}
Пусть функция $x=\phi (t)$ имеет непрерывную производную, тогда
$$
\int f(x)dx = \int f(\phi (t)) \cdot \phi ' (t) dt.
$$

\chapter{Разбиение отрезка, диаметр разбиения.}
Набор $P=\{x_i\}_{i=0}^n \in [a,b]$, где $a=x_0<x_1<...<x_{n-1}<n=b$ называется разбиением отрезка $[a,b]$

Число $\lambda_p = max_{i=1...n} |x_i-x_{i-1}|$ называется диаметром разбиения P.
\chapter{Интегральная сумма Римана.}
Пусть $f:[a,b]\to R$
Пусть $\Delta x_i=x_i-x_{i-1}$

$$\sigma(f;P; \xi)\sum_{i=1}^n f(\xi_i)\Delta x_i\text{ - называется интегральной суммой}$$

\chapter{Определение интеграла Римана по Коши.}
Число $I$ называется интегралом Римана функции f по отрезку $[a;b]$, если $\forall\epsilon > 0 \ \exists \delta > 0: \forall\lambda<\delta \Rightarrow |I - \sigma(f;p;\xi)|<\epsilon$

обозначение:
$$I =\int_a^b f(x)dx$$

\chapter{Аддитивность интеграла Римана по множеству (формулировка).}
$f\in R[a;b],\ c\in(a;b)$, тогда $$\int_a^bf(x)dx = \int_a^cf(x)dx+\int_c^bf(x)dx$$

\chapter{Формула Ньютона-Лейбница.}

Пусть $f\in R[a;b]$ и $F(x)$ - её первообразная, тогда
$$\int_a^bf(x)dx=F(b)-F(a)$$

\chapter{Формула интегрирования по частям в определенном интеграле.}
$$
    \int\limits_{a}^{b} udv = uv |_{a}^{b} - \int\limits_{a}^{b} vdu
$$

\chapter{Замена переменных в определенном интеграле.}
Пусть $\phi: [\alpha;\beta]\to[a;b]$
Причём $\phi$ - непрерывно дифференцируема,
$\phi'(t)$ - непрерывна на $[a;b]$,
$\phi(\alpha) = a; \ \phi(\beta) = b$,
Тогда $\forall$ непрерывной на $[a;b]$ функции:
$$\int_a^bf(x)dx=\int_\alpha^\beta f(\phi(t)) *\phi'(t)dt$$

\chapter{Определение несобственного интеграла Римана по неограниченному множеству.}
Пусть $\forall b \in (a,+\infty)$

$f \in R[a,b]$

Тогда

Если $\exists$ конечный $\lim_{b\to+\infty}\int_a^bf(x)dx=\int_a^{+\infty}f(x)dx$

\chapter{Определение несобственного интеграла Римана по отрезку.}
Пусть $\forall b \in(a,B)$, где $B$ конечно

$f\in R[a,b]$

Если $\exists$ конечный $\lim_{b\to B-}\int_a^bf(x)dx=\int_a^Bf(x)dx$
\end{document}
